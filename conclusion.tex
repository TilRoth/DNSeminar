\section{Conclusion}
\label{sec:conclusion}

The results of the discussed work show that the GDPR had a global impact on privacy control. It helps users be more
informed about privacy and gives them more control especially through third-party opt-out services. On the other hand
the results propose that current state of tracking is still far from the desired behavior. Many websites do not follow
the GDPR by tracking without consent (more than 90\% of the larger websites) and making cookie consent notices
intentionally complicated.

Other work in this area shows similar results. The consensus is that privacy control on
websites is far from perfect. Privacy policies are too long and complicated for people to actually read or understand
them. Cookie consent notices incorporate dark patterns in order to get people to consent to tracking via cookies and still
track people even if they do not consent. Websites are missing the incentive to make privacy control transparent and
accessible.

There is still work to be done on the topic through new regulations, better technical means for detecting violations,
better technical means for opting-out and an incentive for websites to follow the guidelines of these regulations. The
blog post of \ca{whathappened} proposes to introduce a machine-readable form of the GDPR in order to be able to enact
the law via automated control. \ca{degeling2018we} state that it should be made more clear which cookies can be seen as
"legitimate interest" and which should require consent.

