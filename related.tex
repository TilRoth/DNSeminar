\section{Related Work}
\label{sec:related}

In this section we highlight related work in the field. There are many papers regarding the GDPR on the impact or
the legal compliance of websites. We show the methodology and results of related papers and put the underlying
work~\cite{sanchez2019can} into context.

In 2018 \ca{degeling2018we} conducted a similar study to the one of \ca{sanchez2019can}. They evaluate the impact of the
GDPR on cookie consent notices and privacy policies. They only target websites inside the EU, and their domain are the
Alexa top-1M websites. The data collection for the cookie consent notices is manual meaning that
\citeauthor{degeling2018we} went through the websites of their domain by hand, and for privacy policies it is
semi-automated meaning that they tried to collect every privacy policy automated and if there was an error, they tried
it by hand. \citeauthor{degeling2018we} conclude that after the GDPR there are more and more readable privacy policies
and there are more cookie consent notices. However, they also say that these cookie notices may give users a false sense
of privacy and security since they often only inform the user that they are being tracked without the means to prevent
that. Users do not have more privacy online by only being informed that they are being tracked. The work of
\ca{sanchez2019can} follows a very similar methodology and reach a similar conclusion. They extend the work of
\ca{degeling2018we} by also taking international websites into account and observing the actual tracking that is being
performed after navigating through the cookie consent notice.

The \emph{whotracks.me} database released a blog post~\cite{whathappened} where they discuss the impact of the GDPR.
This database is based on real user data creating a realistic image of the situation. They compare USA-based websites to
websites inside the EU in terms of cookie tracking. The method for identification of tracking cookies is through known third-party
tracker lists. This method has the caveat of missing first-party tracking. Also, they do not factor in the consent of
users on tracking. The conclusion is that the average number of tracking cookies decreased in the EU and increased in
the USA. Cookie consent notices incorporate dark patterns manipulating users decisions on cookie control. The GDPR has
made the tracking market more concentrated meaning that smaller companies lose market share while bigger companies get
more powerful. To fix this problem, the post suggests having a machine-readable form of the GDPR such that the law can
not only be enacted through humans.

