\section{Underlying Work}
\label{sec:underlying}

In this section we discuss the underlying work by \ca{sanchez2019can}. We talk about the methodology they use and the
results they achieved with their strategy.

The main focus of the underlying work is not to evaluate whether websites comply with the law, but rather to evaluate
the impact of the GDPR on the ability of a user to control their privacy online.

\subsection{Methodology}

The domain of their analysis are the Alexa top-1M websites. This is a web service by amazon which lists the websites with
the most traffic according to the Alexa Traffic Rank. From this list, they look at 2000 websites across different
categories from some of the top traffic websites.

The data collection was done manually. This means that \citeauthor{sanchez2019can} went through all the 2000 websites by
hand and noted features of the website. By doing this manually, you are less likely to miss any cookie notice or to
click something wrong. Cookie consent notices can differ tremendously from one website to another making it hard to
automate this task. On the other hand, collecting the data manually makes it harder to recreate the data they produced
and harder to look at different features afterwards because you do not want to manually go through the 2000 websites
again. Additionally, metrics like the waiting time until you decline the cookies or the waiting time until you reload
the page after declining cannot be captured by manual analysis. Manual analysis provides better error resistance but is
less flexible and practical than automated analysis. The data they collect are the cookies set before interacting with the
consent notice, the type of consent notice, the privacy policy and the cookies set after rejecting trackers if possible.
The different types of consent notices are:
\begin{enumerate}[a)]
    \item \emph{Anyway}: users are just informed that they are being tracked
    \item \emph{AutoAccept}: users are informed that by continuing to browse they accept tracking
    \item \emph{OnlyAccept}: there is only an accept button
    \item \emph{AcceptReject}: there is an accept and a reject button
    \item \emph{JustSettings}: there is a more complex settings dialog
\end{enumerate}
Cookie consent notices can be blocking and non-blocking meaning that you can access the website while the consent notice pops up for
non-blocking, and you cannot access the website until you have made a choice for blocking cookie consent notices. For
websites which link to third-party opt-out services(e.g. youronlinechoices), they used the service and observed the
cookie behavior.

The identification of tracking cookies was done through measuring the entropy of the value of a cookie with
\texttt{zxcvbn} \cite{wheeler2016zxcvbn}. If the entropy is high enough to distinguish you from 1 Billion people, they
consider it a tracking cookie. This is a rather new approach to the problem of identifying tracking cookies.
On one hand you can detect cookies which with other methods could have been missed, but on the other hand you can get false positives
where the cookie was not a tracker, but it was still classified as one. The latter is not a problem since the
GDPR specifies that cookies which carry information that in theory could identify you still require consent even if they
are not used as tracking cookies.
A more common approach is to make use of lists of known third-party trackers. This approach cannot reliably identify
first-party trackers and can miss some unknown third-parties. \citeauthor{sanchez2019can} use third-party tracker lists
to classify their found trackers.

{\color{red}TODO privacy policies FRES score}

\subsection{Evaluation}

\subsubsection{Tracking}

\subsubsection{Privacy Policies}

