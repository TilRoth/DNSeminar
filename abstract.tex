\begin{abstract}
On May 25th, 2018 the \emph{General Data Protection Regulation} (GDPR) was applied by the European Union. The GDPR is
supposed to provide users the means to protect their privacy on the internet. This also strongly limits the ability of
websites to track users via cookies.

Following the GDPR there was a huge wave of researchers evaluating the effect of this regulation and how well it was
implemented. This report is based on the work of
\ca{sanchez2019can} who evaluate the impact of the GDPR on cookie
tracking of users on a global scale. Furthermore, \citeauthor{sanchez2019can} present the impact of the GDPR on privacy
policies of websites.

They conclude that the GDPR has impacted websites in both tracking and privacy policies. Opting out through third-party
services is helping to reduce the amount of tracking on websites. Privacy policies have got more readable due to the
GDPR and more websites actually display a privacy policy. They also show that tracking is still ubiquitous. More than
90\% of websites still deploy tracking cookies even though the user did not give consent often rendering the cookie
consent notices where you can click \emph{Reject all non-essential cookies} as misleading.
\end{abstract}

