\begin{abstract}
On May 25th, 2018 the \emph{General Data Protection Regulation} (GDPR) was applied by the European Union. The GDPR is
supposed to provide users the means to protect their privacy on the internet. This strongly limits the ability of
websites to track users via cookies because the GDPR specifies that users must not be tracked if they do not consent to
it. Following the GDPR there was a wave of researchers evaluating the effect of this regulation and how well it was
implemented. Other work in this area evaluates the impact of the GDPR on EU websites. We extend this work by evaluating 
the impact of the GDPR on cookie tracking of users on a global scale. Furthermore, we present
the impact of the GDPR on the readability of privacy policies of websites.

We conclude that the GDPR has impacted websites on a global scale. USA-based
websites and EU-based websites were affected similarly. We show that tracking is still ubiquitous. More than
90\% of websites still deploy tracking cookies even though the user did not give consent. Opting-out through third-party
services can help to reduce tracking on up to 70\% of the websites including those that do not follow the guidelines of
the GDPR.
\end{abstract}

