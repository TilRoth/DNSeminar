\section{Introduction}
\label{sec:intro}

The \emph{General Data Protection Regulation} (GDPR) was issued on the 25th of May by the European Union. The goal is to
have a clear regulation on how websites need to handle user data. Almost every website performs some kind of tracking on
users and the GDPR wants to clarify that this tracking without any consent is illegal because tracking poses a risk on
privacy since you can reconstruct a users browsing history. As a result of this many
websites updated their privacy policies and users were flooded with emails. Cookie consent notices also have become more
common where websites first ask you whether you accept or decline non-essential cookies (i.e. tracking or analytical
cookies).

One may think that there is less tracking now due to the consent notices, but the reality is that most websites still
deploy tracking cookies even if you decline. Personalized advertisement is a huge market and most websites get a large
part of their revenue through personalized advertisements. \ca{utz2019informed} propose that less than 0.1\% of users
would accept being tracked if the consent notice has a clear \emph{Accept} and \emph{Decline} button. For websites, this
means that a large part of their revenue would just be lost. The fines they face if they violate the GDPR are smaller
than the revenue they would otherwise lose. This incentivizes misleading cookie consent notices and dark patterns like
hiding the \emph{Decline} option behind different layers, having to unmark already marked checkboxes and performing
tracking without consent.

The main focus of papers in this area is to evaluate whether and how accurately websites follow the guidelines of the
GDPR and in how far the GDPR helps users to have more control over their privacy on the internet.
In the remainder of this report we will discuss the contributions of \ca{sanchez2019can} and put them into context
with related work in this area.

