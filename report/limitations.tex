\section{Limitations}
\label{sec:limitations}

In this section we briefly highlight some limitations of the underlying work.

In \autoref{subsec:methodology} we talk about the methodology of \ca{sanchez2019can}. They perform a manual analysis of
the websites in their target list. A drawback of this method is that specific metrics like the waiting time until you
decline the cookies or the waiting time until you reload the page after declining cannot be captured by manual analysis.
Another drawback is the limitation of target list size. Automated approaches can search through hundreds of thousands
websites where the manual analysis is strongly limited. Additionally, the results of the manual analysis are less
feasible to reproduce and there might be more inconsistencies compared to an automated approach.

In \autoref{subsec:eval} we show the results of the underlying work~\cite{sanchez2019can}.
A critique point here is that the evaluation on cookie consent notices and tracking of \citeauthor{sanchez2019can}
is lacking a proper comparison of the time before
the GDPR and the time after. They only show their results in a post GDPR world leading to some statements being out of
context (e.g. There are no numbers on how many cookie consent notices there were in the EU before the GDPR).

Another shortcoming is the analysis of the impact of the GDPR on e.g. users from the USA on websites based in
the USA. So far in this research area the impact of the GDPR has only been evaluated on EU users visiting international websites and
international users visiting EU-based websites but not on international users visiting international websites.

