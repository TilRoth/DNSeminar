\section{Technical Background}
\label{sec:background}

In this section we give a basic introduction into the techniques used for tracking on websites, namely cookies and
fingerprinting. Cookies are more commonly used as a tracking mechanism than fingerprinting and are also the focus of
the underlying work. Afterwards, we highlight
the core aspects of the GDPR and how tracking should be handled according to these regulations.

\subsection{Tracking}

\paragraph{HTTP cookies.}

HTTP is a stateless protocol meaning that it cannot remember information from previous HTTP requests. In practice this
can be cumbersome. To make HTTP stateful, Netscape introduced cookies. This initial proposal was then refined by the
\emph{Internet Engineering Task Force} (IETF) \cite{rfc6265}. A cookie is a small piece of data often containing a name,
a value, a corresponding URL and a time to live. The HTTP headers Cookie, Set-Cookie are used by the server and the user agent to
exchange cookies. The HTTP server sets cookies in the browser of the user agent and if the user agent sends another HTTP
request to that server, corresponding cookies are sent along such that the server can identify the user and act accordingly. This
mechanism enables stateful HTTP and common features like "shopping carts" and "stay logged in".

Since cookies can be used to uniquely identify users, companies started using cookies not only to enable useful features
(e.g. shopping carts) but also for tracking users across websites. The collected data of the tracked
users is valuable since it can be used for
personalized advertisement and analytics. The basic idea is that a third-party site sets an identifier cookie on a
user accessing a website. If the identifier then appears on other websites, the third-party can identify the user
through the cookie and can thus reconstruct the browsing history of a user.

\paragraph{Fingerprinting.}

Tracking through fingerprinting is generally harder to implement than cookie tracking but can also be more effective. A company
creates a fingerprint of a user by looking at the combination of different factors that can uniquely identify a user. This can
be e.g. the browser in use, the browser add-ons, hardware in use and preferences. For instance, if a user browses the
web on a new common Laptop with a clean Firefox browser, this user is less
likely to be unique than someone with 20 add-ons and a browser that hardly anyone uses on a custom build PC.
Websites include an arbitrary amount of these factors to make users more unique. It is harder to counter fingerprinting
than cookie tracking.

\subsection{GDPR}

On May 25th, 2018 the \emph{General Data Protection Regulation} (GDPR)~\cite{EUdataregulations2018} was applied by the European Union. It is an EU
legislation that is supposed to provide users the means to protect their privacy on the internet. In this section we try
to outline the main ideas of the GDPR concerned with cookie tracking of users and privacy policies.

\textbf{The GDPR applies to user data which can be attributed to an \emph{identifiable natural person}.}
Anonymous data is not considered. Pseudonymous data however is also seen as identifiable because with additional information
the data could be de-anonymized. This implies that not only the tracking itself is illegal but also the collecting of data
which could in theory be used to track a user. This rule specifically targets tracking cookies since their purpose is
solely to uniquely identify a user. The GDPR does not differentiate between third-party or first-party trackers.
Generally every cookie with a large enough entropy (value is so long that it is pretty much impossible that it is not
unique) is affected by this part of the GDPR because it could in theory be used to track a user.

\textbf{The GDPR impacts the handling of user data by websites inside and outside the EU.} This is the first
aspect that is of interest to us. The GDPR not only restricts websites in the EU but has an extraterritorial scope.
It specifies that not only EU websites need to follow the guidelines for handling user data but also websites not in the
EU processing data of citizens of the EU.

\textbf{Personal data should not be stored longer than necessary.} The GDPR specifies that the time of holding personal data of
users should be kept to a minimum. This is usually 12 months.

\textbf{User consent must be explicit.} Websites often give you a notice, that the website uses cookies and an "OK" button
indicating that if a user continues browsing, he accepts cookies (implicit consent). The GDPR states
that websites must not collect personal data before the user explicitly accepts it. Some websites realize
this by having a popup where a user can accept or decline cookie tracking. You should not
\emph{opt-out} but rather \emph{opt-in}. Often, websites track you until you \emph{opt-out} (revoke consent) violating
the GDPR. An important addition: Websites must not refuse to serve a user because he does not give consent. Without this
addition, the explicit consent would not be free, as the user has no choice if access to the website is required.

\textbf{Privacy policies should be transparent and easily accessible.} Often, websites phrase their privacy policy
extra complicated, so people do not bother to read it or do not understand it. The GDPR specifies that the privacy policy
should be expressed in a clear language such that everyone can understand it.

